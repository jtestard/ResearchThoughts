\documentclass[11pt]{article}

\usepackage[utf8]{inputenc}
\usepackage{amsmath}
\usepackage{amssymb}
\usepackage{parskip}
\usepackage{graphicx}
\usepackage{float}
\restylefloat{table}
\usepackage{algorithm2e}

% Relational algebra symbols from ftp://reports.stanford.edu/www/dbgroup_only/latex-macros.html
\newcommand{\select}{\mbox{\Large$\sigma$}}
\newcommand{\cross}{\mbox{$\times$}}
\newcommand{\intersection}{\mbox{$\cap$}}
\newcommand{\intersect}{\mbox{$\cap$}}
\newcommand{\union}{\mbox{$\cup$}}
\newcommand{\join}{\mbox{$\Join$}}
\newcommand{\leftsemijoin}{\mbox{$\mathrel{\raise1pt\hbox{\vrule height5pt
depth0pt width0.6pt\hskip-1.5pt$>$\hskip -2.5pt$<$}}$}}
\newcommand{\rightsemijoin}{\mbox{$\mathrel{\raise1pt\hbox{\hskip-1.5pt$>$\hskip -2.5pt$<$\hskip -1.1pt\vrule height5pt
depth0pt width0.6pt}}$}}
\newcommand{\project}{\mbox{\Large$\pi$}}
\newcommand{\Project}{\mbox{$\Pi$}}
\newcommand{\aggregatefn}{\mbox{\Large$G$}}

\begin{document}
\section*{Observations}

\begin{itemize}
\item{If $A \join B$ is condensing/expansive, then so is $A \leftsemijoin B$.}
\item{We assume identical operators (identical type with identical configuration) have identical cost across sites.}
\end{itemize}

Given two relations $A$ and $B$, an aggregation operator $\gamma_{i,j \rightarrow k}$ where $i$ is the set of grouping attributes, $j$ is the set of aggregated attributes and $k$ is the set of output attributes, and some join operator $\join_D$ for with join predicates $D$. One and only one of the following has to be true in a query plan involving $A$,$B$,$\gamma$ and $\join$:
 \begin{enumerate}
\item{$\project_{i,k}(\gamma_{i,j \rightarrow k}(A) \join_D B) = \gamma(A \join B) $ This happens if $i$ and $j$'s predicates only depend on a single relation ($A$ or $B$) and attributes from $k$ are not used in $D$ and all attributes in $D$ are distinct [and maybe more conditions].}
\item{ $\gamma_{i,j \rightarrow k} (A \join_D B)$. This happens if $i$ and $j$'s predicates depend on both relations or if $D$ depends on an attribute in $j$.}
\item{$\gamma_{i,j \rightarrow k}(A) \join_D B$. This happens if $i$ and $j$'s predicates only depend on a single relation $A$ and $D$ does not depend on an attribute in $j$.}
\item{$\gamma_{i,j \rightarrow k}(A) \join_D \gamma_{i',j' \rightarrow k'}(B)$. This happens if $i/i'$ and $j/j'$'s predicates only depend on a single relation $A/B$ and $D$ does not depend on an attribute in $j$ or $j'$.}
\end{enumerate}

Case 1 is a very restricted class of queries and maybe the class is even smaller than I believe it is (I haven't found exceptions yet, but there might be some). Therefore I do not consider in what follows next.

\pagebreak
\section*{Secenario}

\pagebreak
\section*{Equivalence Rules}

Assuming notation previously introduced and the use of primary keys for $A$ and $B$, here is the equivalence rule used for the "classical" semi join reduction:
\begin{align*}
A \join_D B & = A \join_C ( B \leftsemijoin_D \project_C(A) )\\
& = (A \leftsemijoin_D \project_{C'}(B)) \join_D B\\
\end{align*}
where $C$ is the list of attributes of $A$ which appear in $D$ and $C'$ the list of attributes in $B$ which appear in $D$. Note that commutative equivalences are not shown here but they exist given the commutativity of the join operator.\\

Now if we add aggregation, we have two extra equivalence rules :
\begin{itemize}
\item{ All attributes in predicates in $i,j$ and $D$ which refer to attributes in relation $A$ have to be present in $C$. 
\begin{align*}
\gamma_{i,j \rightarrow k}(A) \join_D B = (\gamma_{i,j \rightarrow k}(A) \leftsemijoin_D  \project_C(B)) \join_D B\\
\end{align*}}
\item{All attributes in predicates in $i/i',j/j'$ and $D$ which refer to attributes in relation $A/B$ have to be present in $C$.
\begin{align*}
\gamma_{i,j \rightarrow k}(A) \join_D \gamma_{i',j' \rightarrow k'}(B) = (\gamma_{i,j \rightarrow k}(A) \leftsemijoin_D  \project_C(\gamma_{i',j' \rightarrow k'}(B))) \join_D \gamma_{i',j' \rightarrow k'}(B) \\
\end{align*}}
\end{itemize}

\pagebreak
\section*{Semi-join process with aggregation}

\subsection*{Usefulness}
The use of semi-join reduction is useful when, assuming $|A| < |B|$ :
\begin{align*}
ec(A) + ec( R = A \join B) + ec(\gamma(R)) & \ge \\
ec(S = \project(A)) + tc(S) + ec (S' = S \leftsemijoin B) & + ec (S'' = \gamma(S')) + tc(S'') + ec(A \join S'')
\end{align*}
where $ec(R)$ is the execution cost of an operator and $tc(R)$ is the transfer cost of the output table of the operator.
\pagebreak


\section*{Set Vs Bag Semantics}
The problem of set vs bag semantics asks itself in the context of semi join reduction when joining back with the relation $A$ on site 1. When using set semantics with primary keys for all relations, the primary key should be projected along other required attributes of $A$ in the first step of the semi-join reduction process. When using bag semantics, 



\end{document}